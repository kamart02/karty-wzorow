% Copyright (C) 2016  Mateusz Piotrowski
%
% This program is free software: you can redistribute it and/or modify
% it under the terms of the GNU General Public License as published by
% the Free Software Foundation, either version 3 of the License, or
% (at your option) any later version.
%
% This program is distributed in the hope that it will be useful,
% but WITHOUT ANY WARRANTY; without even the implied warranty of
% MERCHANTABILITY or FITNESS FOR A PARTICULAR PURPOSE.  See the
% GNU General Public License for more details.
%
% You should have received a copy of the GNU General Public License
% along with this program.  If not, see <http://www.gnu.org/licenses/>.

\documentclass[10pt,a4paper,twocolumn]{article}

\usepackage{multicol}
\usepackage{amsbsy, amssymb, latexsym, amsmath, braket}
\usepackage[tiny]{titlesec}
\usepackage[hmargin=0.5cm,vmargin=1.0cm]{geometry}
\usepackage[utf8x]{inputenc}
\usepackage{polski}
\usepackage{scalefnt}
\usepackage[yyyymmdd,hhmmss]{datetime}

% Potrzebne do algorytmu Euklidesa.
\usepackage{tikz}
\usetikzlibrary{tikzmark}

\newcommand{\angles}[1]{\left\langle #1 \right\rangle}

\titlespacing{\section}{0pt}{0pt}{0pt}
\titlespacing{\subsection}{0pt}{0pt}{0pt}
\titlespacing{\subsubsection}{0pt}{0pt}{0pt}

\setlength{\parindent}{0pt}
% Odległość pomiędzy liniami. Zmniejsz, jeżeli brakuje miejsca.
\setlength{\parskip}{0.5ex}

\title{Karta wzorów z matematyki dyskretnej}

\begin{document}
% Rozmiar czcionki.
\scalefont{.8}

\text{\tiny{
    Wersja z \today\ o \currenttime\ (\pdfmdfivesum file{./karta-wzorow.tex})
}}

\section{Sumy}

${S_n = \sum^{n}_{k = 0} a \cdot x^k} = {a \frac{1 - x^{n+1}}{1 - x}}$;

\subsection{Sumowanie przez części}

${\Delta f(x) = f(x + 1) - f(x)}$;
${\Delta x^{\underline{n}}} = {n x^{\underline{n - 1}}}$;
${\mathcal{S}^{b}_{a} f}={g |^b_a} = {\sum^{b - 1}_{k = a} f(k)}$;
${\Delta (rt)} = {r \Delta t + E t \Delta r \text{, gdzie } E t(x)} =
  {t(x + 1)}$;
${\mathcal{S}^{b}_{a} r \Delta t}= {r t |^b_a - \mathcal{S}^b_a E t \Delta r}$;

\subsection{Dwumian}

${\binom{n}{k} = {\frac{n^{\underline{k}}}{k!}}}$;
${\sum_{i = 0}^n \binom{n}{i} = 2^n}$;
${\sum_{i = 0}^{n} (-1)^i \binom{n}{i} = [ n = 0 ]}$;
$\binom{n}{k} = \binom{n}{n - k}$;

$\binom{n}{k} = \frac{n}{k} \binom{n - 1}{k - 1}$;

${(x + y)^n = \sum_{i \geq 0} \binom{n}{i} x^i y^{n-i}, n \in \mathbb{N}}$;

${(1 + x)^r = \sum_{i \geq 0} \binom{r}{i} x^i, r \in \mathbb{R}, |x| < 1}$;

$(x + y + z)^n =
  \sum_{0 \leq a, b \leq n, a+b \leq n} \binom{n}{a,b} x^a y^b z^{n-a-b}$;

$\binom{n}{a,b} = \binom{n}{a}\binom{n-a}{b} = \frac{n!}{a! b! (n - a - b)!}$;

$\binom{n}{k}\binom{k}{i} = \binom{n}{i}\binom{n-i}{k-i}$;

$\binom{n}{k} = \begin{cases}
    1, k = 0 \text{ lub } k = n \\
    \binom{n - 1}{k} + \binom{n - 1}{k - 1}, 0 < k < n
\end{cases}
$;

% Tożsamość Cauchy'ego.
Toż. (Cauchy): $\sum^k_{i=0}\binom{n}{i}\binom{m}{k - i} = \binom{n + m}{k}$;

$\text{$\sum$ równoległe: } \sum^k_{i=0} \binom{y + i}{i} =
  \binom{y + k + 1}{k}$;

$(-1)^i\binom{x}{i} = \binom{i - 1 - x}{i} \text{, bo }  x^{\underline{i}} =
  (-1)^i(i - 1 - x)^{\underline{i}}$;

$a_n = \sum_i\binom{n}{i}(-1)^i b_i \iff b_n = \dots$;

% Inwersje.
\subsection{Inwersje}

$\Pi = 1^{\lambda_1}2^{\lambda_2}\dots n^{\lambda_n}$, to $sgn(\Pi) =
  (-1)^{\Sigma_i\lambda_{2i}}$;

% Liczby Stirlinga I rodzaju.
\subsection{Liczby Stirlinga}

${n \brack k}$ - kolejność elementów cyklu istotna z dokładnością do cyklicznego
  przesunięcia.

${n \brack 0} = [ n = 0 ]$;
${n \brack 1} = (n - 1)!, n > 0$;
$\sum_k{n \brack k} = n!$;

${n \brack k} = (n-1){n-1 \brack k}+{n-1 \brack k-1}$ dla ${k > 0}$;

% Liczby Stirlinga II rodzaju.
${n \brace k}$ - podziały n-zbioru na k bloków;

${n \brace 0} = [n = 0]$;
${n \brace 1} = 1$;
${n \brace 2} = 2^{n-1}-1$;

${n \brace k} = k{n-1 \brace k} + {n-1 \brace k-1}$ dla ${k > 0}$;

% Własności liczb Stirlinga.
$x^n = \sum_k{n \brace k} x^{\underline{k}}$;

$x^{\overline{n}} = \sum_k{n \brack k}x^k$;

$(-x)^{\overline{k}} = (-1)^kx^{\overline{k}}$;

$x^n = \sum_k {n \brace k} (-1)^{n-k} x^{\overline{k}}$;

$x^{\underline{n}} = \sum_k {n \brack k} (-1)^{n-k} x^{k}$;

$x^n = \sum_{i,k} {n \brace i}{i \brack k}(-1)^{n-i}x^k$;

$\sum_i{n \brace i}{i \brack k}(-1)^{n-i} = [n=k] =
  \sum_i{n \brack i}{i \brace k}(-1)^{n-i}$;

$a_n = \sum_i{n \brace i}(-1)^ib_i \iff b_n = \sum_i{n \brack i}(-1)^ia_i$;

% Funkcje tworzące.
\section{Funkcje tworzące}

$\alpha A(x) + \beta B(x) \leftrightsquigarrow \angles{\alpha a_n + \beta b_n}$;

$x^mA(x) \leftrightsquigarrow
  \angles{\underbrace{0, \dots, 0}_m, a_0, a_1, \dots}$;

$\frac{A(x) - \sum_{i=0}^{m-1}a_ix^i}{x^m} \leftrightsquigarrow
  \angles{a_m, a_{m+1}, \dots}$;

$A(x) \cdot B(x) =
  \sum_n \sum_k a_kb_{n-k}x^n \leftrightsquigarrow
  \angles{\sum_k a_k b_{n-k}}_n$;

$A'(x) = a_1 + 2a_2x + \dots \leftrightsquigarrow = \angles{(n+1)a_{n+1}}_n$;

$\int_0^xA(t)dt \leftrightsquigarrow \angles{\frac{a_{n-1}}{n}}_{n \geq 1}$;

\subsection{Przykłady funkcji tworzących}

$\angles{1,1,\dots} \leftrightsquigarrow \frac{1}{1-x}$;

$\angles{a_n} \cdot \angles{1,1,\dots} = \angles{a_0 + \dots + a_n}_n$;

$\frac{1}{(1-x)^k} \leftrightsquigarrow \angles{\binom{n + k -1}{k - 1}}_n$;

$ln \frac{1}{1-x} \leftrightsquigarrow
  \angles{0, 1, \frac{1}{2}, \frac{1}{3}, \dots}$;

\subsection{Wykładnicze funkcje tworzące}

$B_e(x) = \sum_n \frac{b_nx^n}{n!}$;

$A_e(x) \cdot B_e(x) =
  \sum \left ( \sum_k \binom{n}{k} a_k b_{n-k} \right ) x^n / n!$;

$B'e(x) = \sum_n b_{n+1} x^n/n!$;

$\int^x_0 \sum_{n \geq 0} b_nt^ndt = \sum_{n \geq 1} b_{n-1}x^n/n!$;

$\frac{C}{(1-\lambda x)^k} =
  \sum_{n\geq 0} C \binom{n + k - 1}{k - 1} \lambda^n x^n$;

% TODO: Dodaj przykład rozwiązania równania rekurencyjnego przy pomocy f.t.


Proste ciągi i ich funkcje tworzące:
\begin{tabular}{ | l | l | l |  }
    \hline
    $<1,0,0,\dots>$
      & $\sum_{n \geq 0}[n=0] z^n$
      & $1$ \\
    $<0,\dots, 0,1,0,\dots>$
      & $\sum_{n \geq 0}[n=m] z^n$
      & $z^m$ \\
    $<1,1,1,1>$
      & $\sum_{n \geq 0} z^n$
      & $\frac{1}{1-z}$ \\
    $<1,-1,1,-1>$
      & $\sum_{n \geq 0} (-z)^n$
      & $\frac{1}{1+z}$ \\
    co $m$-te
      & $\sum_{n \geq 0} [m | n]z^n$
      & $\frac{1}{1-z^m}$ \\
    $<1,2,3,4>$
      & $\sum_{n \geq 0} (n+1)z^n$
      & $\frac{1}{(1-z)^2}$ \\
    $<1,c,\frac{c}{2},\frac{c}{2}>$
      & $\sum_{n \geq 0} \binom{c}{n}z^n$
      & $(1+z)^c$ \\
    $<1,c,c^2,c^3>$
      & $\sum_{n \geq 0} c^n z^n$
      & $\frac{1}{1-cz}$ \\
    $<1,\frac{m+1}{m},\frac{m+2}{m},>$
      & $\sum_{n \geq 0} \binom{m+n}{m} z^n$
      & $ \frac{1}{(1-z)^{m+1}}$ \\
    $<0,1,\frac{1}{2},\frac{1}{3}>$
      & $n \geq 1$
      & $\ln \frac{1}{(1-z)}$ \\
    \hline
\end{tabular}

\section{Magiczne liczby}

\begin{tiny}
      2      3      5      7     11     13     17     19     23     29
     31     37     41     43     47     53     59     61     67     71
     73     79     83     89     97    101    103    107    109    113
    127    131    137    139    149    151    157    163    167    173
    179    181    191    193    197    199    211    223    227    229
    233    239    241    251    257    263    269    271    277    281
    283    293    307    311    313    317    331    337    347    349
    353    359    367    373    379    383    389    397    401    409
    419    421    431    433    439    443    449    457    461    463
    467    479    487    491    499    503    509    521    523    541
    547    557    563    569    571    577    587    593    599    601
    607    613    617    619    631    641    643    647    653    659
    661    673    677    683    691    701    709    719    727    733
    739    743    751    757    761    769    773    787    797    809
    811    821    823    827    829    839    853    857    859    863
    877    881    883    887    907    911    919    929    937    941
    947    953    967    971    977    983    991    997   1009   1013
   1019   1021   1031   1033   1039   1049   1051   1061   1063   1069
   1087   1091   1093   1097   1103   1109   1117   1123   1129   1151
   1153   1163   1171   1181   1187   1193   1201   1213   1217   1223
   1229   1231   1237   1249   1259   1277   1279   1283   1289   1291
   1297   1301   1303   1307   1319   1321   1327   1361   1367   1373
   1381   1399   1409   1423   1427   1429   1433   1439   1447   1451
   1453   1459   1471   1481   1483   1487   1489   1493   1499   1511
   1523   1531   1543   1549   1553   1559   1567   1571   1579   1583
   1597   1601   1607   1609   1613   1619   1621   1627   1637   1657
   1663   1667   1669   1693   1697   1699   1709   1721   1723   1733
   1741   1747   1753   1759   1777   1783   1787   1789   1801   1811
   1823   1831   1847   1861   1867   1871   1873   1877   1879   1889
   1901   1907   1913   1931   1933   1949   1951   1973   1979   1987
   1993   1997   1999   2003   2011   2017   2027   2029   2039   2053
   2063   2069   2081   2083   2087   2089   2099   2111   2113   2129
   2131   2137   2141   2143   2153   2161   2179   2203   2207   2213
   2221   2237   2239   2243   2251   2267   2269   2273   2281   2287
   2293   2297   2309   2311   2333   2339   2341   2347   2351   2357
   2371   2377   2381   2383   2389   2393   2399   2411   2417   2423
   2437   2441   2447   2459   2467   2473   2477   2503   2521   2531
   2539   2543   2549   2551   2557   2579   2591   2593   2609   2617
   2621   2633   2647   2657   2659   2663   2671   2677   2683   2687
   2689   2693   2699   2707   2711   2713   2719
\end{tiny}

\subsection{Liczby Catalana}

$C_n$ - liczba nawiasowań w wyr. $x_0 \cdot x_1 \cdot \dots \cdot x_n$;

$C_n = \sum_kC_kC_{n-1-k} + [n=0]$,
  skąd $C(x) = \sum_n C_nx^n = x(C(x))^2 + 1, C(0) = 1$,
  więc $C(x) = \frac{1 - \sqrt{1-4x}}{2x}$, a zatem $C(x) =
  \sum_{k\geq 0} \frac{1}{k+1}\binom{2k}{k}x^k$;

\subsection{Liczby Bella}

${B_n = \sum_k{n \brace k}}$ - łączna l. podziałów n-zbioru na bloki;

${B_{n+1} = \sum_k \binom{n}{k} B_k}$, skąd $B'(x) = e^xB(x), B(0) = 1$,
  gdzie $B(x) = \sum_n B_nx^n/n!$. Całkujemy i dostajemy $ln B = e^x + c$,
  skąd $B(x) = e^{e^x-1} =
  \sum_{n=0}^\infty \left ( {1 \over e} \sum_{k=0}^\infty {k^n \over k!}\right)
  {x^n \over n!}$;

% Enumeratory.
\subsection{Enumeratory kombinacji}

$\underbrace{(1+t)\dots(1+t)}_n = \sum_r\binom{n}{r}t^r$;

Kombinacja z dowolnymi powt.: $(1+t+t^2+\dots)^n=(1-t)^{-n}=
  \sum_r\binom{-n}{r}{-t}^r=\sum_r\binom{n+r-1}{r}t^r$;

Każdy elem. co najmniej raz: $(t+t^2+\dots)^n = t^n\sum_r {n+r-1 \over r}t^r =
  \sum^\infty_{r=n}\binom{r-1}{n-1}t^r$;

\subsection{Enumeratory permutacji}

r-permu. bez powt.: $(1+t)^n = \sum_rn^{\underline{r}}{t^r \over r!}$;

Dow. l. powt.: ${(1+t+t^2/2! + \dots)^n} = {(e^{t})^n} =
  {\sum_rn^r{t^r \over r!}}$;

Każdy elem. co najmniej raz: $(t + t^2/2! + \dots)^n = (e^t - 1)^n =
  \sum^\infty_{r=0}\frac{t^r}{r!}\sum^n_{j=0}\binom{n}{j}(-1)^j(n-j)^r$;

Ciąg $A$, $B$, $C$ długości $r$ t., że $\#A>0$, $2|\#B$. Traktuj jako r-permu.
  z powt.:
  $\overbrace{(e^t - 1)}^A\cdot\overbrace{((e^t - e^{-t})/2)}^B
  \cdot\overbrace{e^t}^C =
  \sum_{r \geq 1} {1 \over 2}(3^r - 2^r - 1){t^r \over r!}$;

\section{Zasada włączania-wyłączania}

Liczba elem. o $j$ własn.: $S_j =
  \sum_{1 \leq i_1 < \dots < i_j \leq n} |A_{i_1} \cap \dots \cap A_{i_j}|$;

Liczba elem. o dokładnie $k$ własn.:
  $D(k) = \sum_{j\geq k} \binom{j}{k} (-1)^{j-k}S_j$;

Zliczanie $n$-nieporządków: $A_i = \left\{n\text{-perm.} | f(i) = i \right\}$.
  Wtedy $|A_{i_1} \cap \dots \cap A_{i_j}| = (n-j)!$, zatem
  $D(0) = \sum^n_j(-1)^j\binom{n}{j}(n-j)!=n!\sum^n_j(-1)^j{1 \over j!}$;

% TODO: Wieżomiany.

\section{Podział liczby / enumerator podziałów}

Podział liczby $n$, to przedstawienie $n$ jako sumy nierosnących dod. skład.;

L. podz. $n$ na $\leq k$ skład. $\leftrightsquigarrow$
  l. podz. $n+k$ na dokł. $k$ skład.;

Enum. podziałów $P(x) = \sum_nP_nx^n =
  (1+x+x^2+\dots)\cdot(1+x^2+x^4+\dots)\dots(1+x^k+\dots)\dots =
  \frac{1}{(1-x)(1-x^2)\dots(1-x^k)\dots}$, gdzie wybranie z $k$-tego nawiasu
  $x^{ik} \leftrightsquigarrow$ wzięciu do podziału $i$ razy składnika $k$;

Enum. podz. o składnikach $\leq k$: $p_{\leq k} (x) =
  \frac{1}{(1-x)(1-x^2)\dots(1-x^k)}$;

Enum. podz. 1 zł na grosze:
  $\left [ x^{100} \right ]
  \frac{1}{(1-x)(1-x^2)(1-x^5)(1-x^{10})(1-x^{20})(1-x^{50})}$;

Enum. podz. na składniki $>1$: $\frac{1}{(1-x^2)(1-x^3)\dots}=(1-x)P(x) =
  P_n - P_{n-1}, n>0$;

Enum. podz. o różnych s.: $r(x) = (1+x)(1+x^2)(1+x^3)\dots$;

Enum. podz. o nieparz. s.: $n(x)=\frac{1}{(1-x)(1-x^3)\dots}$;

Tw.: $\#$ podz. $n$ na różne s. $= \#$ podz. na nieparz. s.: $n(x) = r(x)$;

% TODO: Przykład użycia.
Toż. Eulera: $nP_n = \sum_{k=0}^{n-1}\sigma(n-k)P_k$,
  gdzie $\sigma = \sum_{k|n}k$;

% Grafy.
\section{Grafy}

Lem. o uściskach dł.: $\sum_{v\in V} deg(v) = 2|E|$;

$H$ podgrafem indukowanym $G
  \iff \forall_{u,v \in V[H]} \set{u, v} \in E[G] \implies \set{u, v}\in E[H]$;

Droga nie powtarza krawędzi, a ścieżka wierzchołków;

$G$ spójny $\iff \forall_{u,v\in V[G]} \exists e_{uv}$;

$G$ $k$-reguralny $\iff \forall_v deg(v) = k$;

$G$ dwudzielny, gdy $V[G] = V_1 \cup V_2, V_1 \cap V_2 = \emptyset$ i każda
  krawędź ma jeden koniec w $V_1$, a drugi w $V_2$;

$K_{|V|, |U|}$ pełny dwudzielny, gdy $E=\set{\set{v, u}, v \in V, u \in U}$;

$v$ rozcinający, gdy usunięcie $v$ zwiększa l. spójnych s.;

Tw.: $G$ dwudzielny $\iff$ nie zawiera cykli nieparz. dł.;

\subsection{Cykle}

C. E. - krawędzie; C. H. - wierzchołki; Turniej - skierowana klika;

Tw. Eulera: $G$ ma c. E. $\iff \forall_{v \in V[G]} 2|deg(v)$;

Tw.: Silnie spójny $G$ ma skierowany c. E.
  $\iff \forall_{v \in V[G]} deg_{in}(v) = deg_{out}(v)$;

$G$ ma c. H., to po usunięciu dow. k wierz. rozpada się na co najw.
  k spójnych s.;

$G = \angles{V, U; E}$ dwudzielny ma c. H., to $|V|=|U|$;

Tw.: Każdy turniej jest półhamilton.;

Tw.: Turniej spójny, to ma c. H.;

Tw. (Ore): $n=|V|\geq 3$ i
  $\forall_{\set{v, w} \not\in E} deg(v) + deg(w) \geq n$, to $G$ ma c. H.;

\subsection{Drzewa}

Tw. (Cayley): Jest $n^{n-2}$ etykietowanych drzew $n$-wierzchołkowych;

\subsection{Planarność}

Wz. Eulera: $n - m + f = 2$, gdzie $m = |E[G]|$;

Tw.: W grafie plan. z $n \geq 3$ mamy $m \leq 3n -6$;

Tw. Kuratowskiego: $G$ nieplan. $\iff G$ zawiera podgraf homeomorficzny
  z $K_{3,3}$ lub z $K_5$ (homeomorficzny, czyli
  izomorficzny po ew. dołożeniu wierzchołków na krawędziach);

$G$ planarny $\implies \exists_{v \in G[V]} deg(v) \leq 5$;

\subsection{Kolorowanie wierzchołków}

Kolorowanie $G$ za pomocą $k$ kolorów, to
  $f: V[G] \rightarrow \set{1, \dots, k}$ t., że $f(u) \neq f(v)$ dla
  $\set{u, v} \in E[G]$. Najm. k t., że $\exists k$-kolorowanie $G$, to liczba
  chromatyczna $\chi (G)$.

$\chi(G) \leq 2 \Leftrightarrow G$ dwudzielny;

$\chi(G) \leq k \Leftrightarrow \chi(G) \leq k$ dla każdej dwuspójnej s. $B$
  grafu $G$;

Tw. o 4 barwach: $G$ plan. $\implies\chi(G)\leq 4$;

Tw. Brooksa: $G$ spójny, nie cykl nieparz. dł., nie klika, to
  $\chi(G) \leq \Delta$, gdzie $\Delta$, to maks. stop. wierz. w $G$.;

Tw.: $\chi(G) \leq \Delta + 1$;

$f_G(t)$, to liczba kolorowań $G$ za pomocą $t$ kolorów.

$f_{\overline{K_n}}(t) = t^n$; $f_{K_n} = t^{\underline{n}}$

Tw.: $e = \set{v, w} \not\in E[G]$, to $f_G(t)=f_{G\cup e}(t) + f_{G/e}(t)$.

\subsection{Kolorowanie krawędzi}

Funkcja $f: E[G] \rightarrow \set{1, \dots, k}$, to kolorowanie krawędziowe,
  jeśli kraw. incydentne mają różne kolory. Indeks chromatyczny $\chi_e(G)$, to
  najmniejsze $k$, dla którego istnieje $k$-kolorowanie kraw.

Tw. Vizinga: $\forall_G \chi_e(G) \leq \Delta(G) +1$;

Tw. (K{\"o}nig): $G$ dwudzielny, to $\chi_e(G) = \Delta(G)$;

\subsection{Systemy różnych reprezentantów}

SSR dla rodziny zbiorów $\angles{A_i}_{i\in I},$ to ciąg elem.
  $\angles{a_i}_{i\in I}$ t., że
  $\forall_{i\in I} a_i \in A_i$ oraz $a_i \neq a_j$
  (skojarzenia w g. dwudzielnym);

Tw. (Hall): SRR dla skończonej r. zb. skończonych $\angles{A_i}_{i=1}^n,$
  istnieje
  $\iff \forall_{J\subseteq\set{1,\dots,n}} |\bigcup_{j\in J}A_j| \geq |J|$;

$G$ dwudzielny $r$-regularny $\Rightarrow r$-kolorowalny kraw.;

Tw.: Podziały $\mathcal{A}$ i $\mathcal{B}$ mają wspólny SRR $\Leftrightarrow
  \forall_{J\subseteq I} |\bigcup_{j\in J}g(A_j)| \geq |J|$, gdzie
  $g(C) = \set{j | C \cap B_j \neq \emptyset}$;

$A_1 \cup \dots \cup A_n = B_1 \cup \dots \cup B_n$ i
  $\forall_{1 \leq i \leq n}|A_i| = |B_i| =r \Rightarrow \mathcal{A}$ i
  $\mathcal{B}$ mają SRR;

\section{Teoria liczb}

% NWD.
$NWD(a,b) = min_{+}\set{ax+by | x,y\in \mathbb{Z}}$;

$a \perp b \Leftrightarrow NWD(a,b)=1$;

% Algorytm Euklidesa.
Alg. Euklidesa: $NWD(a,b)=ax+by$. Jeśli $b=0$, to
  $\angles{x,y} \angles{1,0}$, wpp
  $NWD(a,b) = NWD(b, a \mod b) = bx' + (a \mod b)y'$
  ($\angles{x,y} \leftarrow \angles{y', x' - y'\cdot\lfloor{a/b}\rfloor}$);

% Tabelka z przykładem działania algorytmu Euklidesa.
% TODO: Dodaj ramkę.
\begin{tabular}{c c c c}
    20& 56 & 3\tikzmark{aedst3} &-1 \\
    16& 20 & -1\tikzmark{aedst2} &\tikzmark{aesrc3}1 \\
    4 & 16 & 1\tikzmark{aedst1} &\tikzmark{aesrc2}0 \\
    0\tikzmark{aesrc0} & 4 &\tikzmark{aedst0}0 & \tikzmark{aesrc1}1 \\
\end{tabular}
\begin{tikzpicture}[overlay, remember picture, yshift=.25\baselineskip,
    shorten >=.5pt, shorten <=.5pt]
    \draw [->] ({pic cs:aesrc0}) [bend left] to ({pic cs:aedst0});
    \draw [->] ([yshift=.75pt]{pic cs:aesrc1}) -- ({pic cs:aedst1});
    \draw [->] ([yshift=.75pt]{pic cs:aesrc2}) -- ({pic cs:aedst2});
    \draw [->] ([yshift=.75pt]{pic cs:aesrc3}) -- ({pic cs:aedst3});
\end{tikzpicture}

% Podstawowe wiadomości z teorii liczb.
$a = \prod^m_{i=1}p_i$;

${a | bc \land a \perp b} \Rightarrow {a | c}$

${NWW(a,b)} = {ab/NWD(a,b)} = {\prod^k_{i=1}p_i^{max(\alpha_i, \beta_i)}}$;

${a \equiv b \pod{n}} \land {c \equiv d \pmod{n}} \Rightarrow
  {a+c\equiv b+d \pmod{n}} \land {a\cdot c \equiv b\cdot d \pmod{n}}$;

${d \perp n} \land {ad \equiv bd \pmod{n}} \Rightarrow {a \equiv b \pmod{n}}$;

${ad \equiv bd \pmod{nd}} \Leftrightarrow {a\equiv b \pmod{n}}$;

${b = a^{-1} \mod n} \Leftrightarrow {ab \equiv 1 \pmod{n}}$;

${n \perp m} \implies
  {\left( a\equiv b \pmod{n} \land a \equiv b \pmod{m} \Leftrightarrow
  a\equiv b \pmod{nm} \right)}$;

% Chińskie twierdzenie o resztach.
\framebox{\vbox{
    ChTOR: ${n = n_1\dots n_k}, {n_i \perp n_j}$, to
      $\forall_{a_1, \dots, a_k}\exists a\in\set{0,\dots,n-1}$ t., że
      $a\equiv a_i \pmod{n_i}$ dla $i=1,\dots,k$;

    % Przykład użycia chińskiego twierdzenia o resztach.
    % Źródło: wazniak.mimuw.edu.pl:Matematyka dyskretna 1: Wykład 11
    Układ: $x \equiv_3 2$, $x \equiv_10 7$, $x \equiv_11 10$, $x \equiv_7 1$.
      Spr. warunków ChTOR:
      $NWD(3,10)=NWD(3,11)=NWD(3,7)=NWD(10,11)=NWD(10,7)=NWD(11,7)=1$.
      $N = 3\cdot 7\cdot 10\cdot 11 = 2310$,
      $N_1 = \frac{2310}{3}=770$,
      $N_2 = \frac{2310}{10}=231$,
      $N_3 = 210$,
      $N_4 = 330$.
      R. alg. Eukl.:
      $NWD(3, 770) = 1 = 257\cdot 3 - 1\cdot 770$, $x_1=-1\equiv_3 2$,
      $NWD(10, 231) = 1 = -23\cdot 10 + 1\cdot 231$, $x_2=1$,
      $NWD(11,210) = -19\cdot 11 + 1 \cdot 210$, $x_3 = 1$,
      $NWD(7,330) = -47\cdot 7 + 1\cdot 330$, $x_4 = 1$.
      Zatem $x = 2\cdot 2 \cdot 770 + 7\cdot 1\cdot 231 + 10\cdot 1\cdot 210 +
      1\cdot 1\cdot 330$;
}}

% Małe twierdzenie Fermata.
MTF: ${p\in \mathbb{P} \land p\not | a} \Rightarrow a^{p-1} \equiv 1 \pmod{p}$,
  inaczej $a^p \equiv a \pmod{p}$;

% Funkcja Eulera.
F. Eulera: $\phi: \mathbb{N} \rightarrow \mathbb{N}$ t., że
  $\phi(n)=|\mathbb{Z}^*_n|=|\set{1\leq k\leq n : k \perp n}|$;

$p \in \mathbb{P} \Rightarrow \phi(p^k)=p^k-p^{k-1}$;

$m\perp n \Rightarrow \phi(mn) = \phi(m)\phi(n)$;

$\phi(n) = n\prod_{\mathbb{P} \ni p | n}(1- 1/p)$;

$\sum_{d|m}\phi(d) = m$;

% Twierdzenie Eulera.
Tw. Eulera: $a\perp n \Rightarrow a^{\phi(n)} \equiv 1 \pmod{n}$;

% Twierdzenie Wilsona.

Tw. Wilsona: $(p-1)! \equiv -1 \pmod{p}$;

% TODO: RSA.

% Test Millera-Rabina.
Test Millera-Rabina: $\exists_{0<a<n}a^{n-1}\not\equiv 1 \pmod{n} \Rightarrow
  n \not\in \mathbb{P}$ (słaby, bo liczby Carmichaela);

% TODO: Opis liczb Carmichaela.

% TODO: Faktoryzajca metodą Fermata.

% TODO: Faktoryzacja metodą "p-1" Pollarda.

\section{Teoria grup}

$G = \angles{e, \cdot}$; łączność, elem. neutr., odwrotność, przemienność*;

Podgrupa grupy $G$, to podzb. zamkn. na $\cdot$ i $^{-1}$;

Podgr. generowana przez $A$, to przecięcie wszystkich podgr. $G \supseteq A$;

$G(A)$ składa się ze skończonych iloczynów $g_1\dots g_k$,
  $g_i\in A \lor g_i^{-1} \in A$;

Grupa cykliczna, czyli gen. przez 1 elem.;

% Twierdzenie (Lagrange).
Tw. (Lagrange): Rząd podgr. $H$ gr. skończonej $G$ dzieli rząd $G$;

$G$ cykliczna ma $\phi(n)$ gen;

Tw.: $s(d)$, to l. elem. rzędu $d$ w $G$. $s(d) =$ if $d \not{|} n$ then $0$
  else $\phi(d)$;

% Twierdzenie (Cayley).
Tw. (Cayley): $S(Y)$, to zbiór bijekcji $Y \rightarrow Y$ ze składaniem
  ($S_n$ dla $Y=\set{1,\dots,n}$). Każda gr. rzędu $n$ jest izomorf. z pewną
  podgr.  $S_n$;

% TODO: Dokończyć teorię grup. Slajd 216.

\end{document}
