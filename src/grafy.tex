\section{Grafy}

\entry
Lem. o uściskach dł.: $\sum_{v\in V} deg(v) = 2|E|$;

\entry
$H$ podgrafem indukowanym $G
  \iff \forall_{u,v \in V[H]} \set{u, v} \in E[G] \implies \set{u, v}\in E[H]$;

\entry
Droga nie powtarza krawędzi, a ścieżka wierzchołków;

\entry
$G$ spójny $\iff \forall_{u,v\in V[G]} \exists e_{uv}$;

\entry
$G$ $k$-reguralny $\iff \forall_v deg(v) = k$;

\entry
$G$ dwudzielny, gdy $V[G] = V_1 \cup V_2, V_1 \cap V_2 = \emptyset$ i każda
  krawędź ma jeden koniec w $V_1$, a drugi w $V_2$;

\entry
$K_{|V|, |U|}$ pełny dwudzielny, gdy $E=\set{\set{v, u}, v \in V, u \in U}$;

\entry
$v$ rozcinający, gdy usunięcie $v$ zwiększa l. spójnych s.;

\entry
Tw.: $G$ dwudzielny $\iff$ nie zawiera cykli nieparz. dł.;

\subsection{Cykle}

\entry
C. E. --- krawędzie; C. H. --- wierzchołki; Graf h. --- graf z c. H.;

\entry
Tw. Eulera: $G$ ma c. E. $\iff \forall_{v \in V[G]} 2|deg(v)$;

\entry
Tw.: Silnie spójny $G$ ma skierowany c. E.
  $\iff \forall_{v \in V[G]} deg_{in}(v) = deg_{out}(v)$;

\entry
$G$ ma c. H., to po usunięciu dow. k wierz. rozpada się na co najw.
  k spójnych s.;

\entry
$G = \angles{V, U; E}$ dwudzielny ma c. H., to $|V|=|U|$;

\entry
Tw.: Każdy turniej jest półhamilton. (zawiera ś. H.);

\entry
Tw.: Turniej ma c. H. $\iff$ jest silnie spójny;

\entry
Tw.: Turniej spójny, to ma c. H.;

\entry
Tw. (Ore): $n=|V|\geq 3$ i
  $\forall_{\set{v, w} \not\in E} deg(v) + deg(w) \geq n$, to $G$ ma c. H.;

\entry
Każdy turniej hamiltonowski jest silnie spójny;

\entry
$G$ silnie spójny $\Rightarrow$ $G$ zawiera skierowany $k\text{-cykl}$;

\subsection{Drzewa}

\entry
Tw. (Cayley): Jest $n^{n-2}$ etykietowanych drzew $n$-wierzchołkowych;

\entry
Równoważne są:\\
G jest drzewem,\\
każde dwa wierzchołki w G są połączone dokładnie jedną droga,\\
G jest minimalny spójny,\\
G jest maksymalny acykliczny,\\
G jest spójny i $|V| = |E| + 1$ 

\subsection{Planarność}

\entry
Wz. Eulera: $n - m + f = 2$, gdzie $m = |E[G]|$;

\entry
Tw.: W grafie plan. z $n \geq 3$ mamy $m \leq 3n -6$;

\entry
Mocjiejszym twierdzeniem jest gdy graf nie zawiera trójkątów $m \leq 2n - 3$;

\entry
Tw. Kuratowskiego: $G$ nieplan. $\iff G$ zawiera podgraf homeomorficzny
  z $K_{3,3}$ lub z $K_5$ (homeomorficzny, czyli
  izomorficzny po ew. dołożeniu wierzchołków na krawędziach);

\entry
$G$ planarny $\implies \exists_{v \in G[V]} deg(v) \leq 5$;

\subsection{Kolorowanie wierzchołków}

\entry
Kolorowanie $G$ za pomocą $k$ kolorów to
  $f: V[G] \rightarrow \set{1, \dots, k}$ t., że $f(u) \neq f(v)$ dla
  $\set{u, v} \in E[G]$. Najm. k t., że $\exists k$-kolorowanie $G$ to liczba
  chromatyczna $\chi (G)$.

\entry
$\chi(G) \leq 2 \Leftrightarrow G$ dwudzielny;

\entry
$\chi(G) \leq k \Leftrightarrow \chi(B) \leq k$ dla każdej dwuspójnej s. $B$
  grafu $G$;

\entry
Tw. o 4 barwach: $G$ plan. $\implies\chi(G)\leq 4$;

\entry
Tw. Brooksa: $G$ spójny, nie cykl nieparz. dł., nie klika, to
  $\chi(G) \leq \Delta$, gdzie $\Delta$ to maks. stop. wierz. w $G$;

\entry
Tw.: $\chi(G) \leq \Delta + 1$;

\entry
$f_G(t)$ --- wielomian chrom. (liczba kolorowań $G$ za pomocą $t$ kolorów);

\entry
W. ch.:
\entry
$K_n$ --- $t^{\underline{n}}$,
\entry
$\overline{K_n}$ --- $t^n$,
\entry
$\text{Drzewo}_n$ --- $t(t-1)^{n-1}$,
\entry
$\text{Cykl}_n$ --- $(t-1)^n + (-1)^n(t-1)$,
\entry
$K_{n,m}$ --- $\sum_{a,b}{n \brace a}{m \brace b}t^{\underline{a+b}}$;

\entry
Tw.: $e = \set{v, w} \not\in E[G]$, to $f_G(t)=f_{G\cup e}(t) + f_{G/e}(t)$;

\subsection{Kolorowanie krawędzi}

\entry
Funkcja $f: E[G] \rightarrow \set{1, \dots, k}$ to kolorowanie krawędziowe,
  jeśli kraw. incydentne mają różne kolory. Indeks chromatyczny $\chi_e(G)$ to
  najmniejsze $k$, dla którego istnieje $k$-kolorowanie kraw.;

\entry
Tw. Vizinga: $\forall_G \chi_e(G) \leq \Delta(G) +1$;

\entry
Tw. (K{\"o}nig): $G$ dwudzielny, to $\chi_e(G) = \Delta(G)$;

\subsection{Systemy różnych reprezentantów}

\entry
SRR dla rodziny zbiorów $\angles{A_i}_{i\in I},$ to ciąg elem.
  $\angles{a_i}_{i\in I}$ t., że
  $\forall_{i\in I} a_i \in A_i$ oraz $a_i \neq a_j$
  (skojarzenia w g. dwudzielnym);

\entry
Tw. (Hall): SRR dla skończonej r. zb. skończonych $\angles{A_i}_{i=1}^n,$
  istnieje
  $\iff \forall_{J\subseteq\set{1,\dots,n}} |\bigcup_{j\in J}A_j| \geq |J|$;

\entry
$G$ dwudzielny $r$-regularny $\Rightarrow r$-kolorowalny kraw.;

\entry
$G$ dwudzielny, regularny ma pełne skojarzenie;

\entry
$G$ $(n-m)$-regularny $\Rightarrow \exists$ pełne skojarzenie;

\entry
Tw.: Podziały $\mathcal{A}$ i $\mathcal{B}$ mają wspólny SRR $\Leftrightarrow
  \forall_{J\subseteq I} |\bigcup_{j\in J}g(A_j)| \geq |J|$, gdzie
  $g(C) = \set{j | C \cap B_j \neq \emptyset}$;

\entry
$A_1 \cup \dots \cup A_n = B_1 \cup \dots \cup B_n$ i
  $\forall_{1 \leq i \leq n}|A_i| = |B_i| =r \Rightarrow \mathcal{A}$ i
  $\mathcal{B}$ mają SRR;
