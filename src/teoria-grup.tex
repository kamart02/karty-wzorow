\section{Teoria grup}

$G = \angles{e, \cdot}$: łączność, elem. neutr., odwrotność, przemienność*;

Podgrupa grupy $G$, to podzb. zamkn. na $\cdot$ oraz $^{-1}$;

Podgr. generowana przez $A$, to przecięcie wszystkich podgr. $G \supseteq A$;

$G(A)$ składa się ze skończonych iloczynów $g_1\dots g_k$,
  $g_i\in A \lor g_i^{-1} \in A$;

Gr. cykliczna, czyli generowana przez 1 elem.;

Gr. cykliczna $\Leftrightarrow$ przemienna;

Każda podgr. gr. cyklicznej jest cykliczna;

Gr., w której każdy elementy $\neq e$ ma rząd $2$, jest przemienna
  i ma rząd $2^k$;

Rząd elementu $r(g)$ to rząd generowanej przez niego grupy cyklicznej, czyli
najmniejsze $k > 0$ t., że $g\cdot\overset{k}{\ldots}\cdot g=e$;

Warstwa $G$ względem $H$ wyznaczona przez $g\in G$ to
  $gH = \set{gh : h \in H}$. Warstwy są równoliczne z $H$
  ($h \mapsto gh$ są 1--1) i tworzą podział zb. $G$;

Z tego, że $d$ dzieli $|G|$, nie wynika, że $G$ ma podgr. rzędu $d$. Jednak
jeśli $G$ jest cykliczna, to tak jest.

% Twierdzenie (Lagrange).
Tw. (Lagrange): Rząd podgr. $H$ gr. skończonej $G$ dzieli rząd $G$;

$G$ cykliczna ma $\phi(n)$ generatorów;

Tw.: $s(d)$, to l. elem. rzędu $d$ w $G$. $s(d) = 0$ jeżeli $d \not{|} n$
  wpp $\phi(d)$;

% Twierdzenie (Cayley).
Tw. (Cayley): $S(Y)$, to zbiór bijekcji $Y \rightarrow Y$ ze składaniem
  ($S_n$ dla $Y=\set{1,\dots,n}$). Każda gr. rzędu $n$ jest izomorf. z pewną
  podgr. $S_n$;

Działanie $G$ na zb. $X$ --- homomorfizm $h: G \rightarrow S(X)$ (wierne,
jeżeli $h$ jest 1--1). Piszemy $gx$ zamiast $(h(g))(x)$;

Relacja $G\text{-równoważności na } X$:
  $x \sim_G x' \text{, jeśli } \exists_{g\in G} x' = gx$;

Orbita $x \in X$ to jego kl. abstr. $Gx=\set{gx:g\in G}$;

Stabilizator $x \in X$ to podgr. $G_x = \set{g\in G: gx = x}$ grupy $G$;

Zb. punktów stałych $g \in G$ to podzb. $X_g = \set{x\in X: gx=x}$ zbioru $X$;

Tw. (Burnside): L. orbit $=$ średnia liczba punktów stałych
  $=\frac{1}{\abs{G}}\sum_{g\in G}\abs{X_g}$;

Warstwy $G$ względem $G_x \leftrightsquigarrow$ elementry orbity $Gx$;
$\abs{G} = \abs{G_x} \cdot \abs{Gx}$;

Kolorowanie zb. $X$ to $f: X\rightarrow K$, gdzie $K$ to zb. kolorów
  $\abs{K} = k$. Jeśli $G$ działa na $X$ (jest pewną gr. permutacji $X$), to
  rozszerzamy działanie $G$ na zb. kolorowań $F$ wzorem $gf=f\circ g^{-1}$.
  Niech $c(g)$ ozn. liczbę cykli w perm. $g$; Fakt: Liczba nierównoważnych
  kolorowań to $\frac{1}{\abs{G}}\sum_{g\in G}k^{c(g)}$;

Opis struktury cykli w gr. perm. $G \subseteq S_n$, to f. t. od $n$ zmiennych,
  zwana indeksem cyklowym: $I_G(x_1,/ldots,x_n)=
  \frac{1}{\abs{G}}\sum_{g\in G}x_1^{g_1}\cdot\ldots\cdot x_n^{g_n}$, gdzie
  $g_i$ to l. cykli długości $i$ w perm. $g$ (i. c. zależy od działania gr.
  na konkretnym zbiorze i samej grupy). Przykład: liczba nierównoważnych
  kolorowań za pomocą $k$ kolorów to $|_G(k,\ldots,k)$;

Chcemy wyznaczyć liczbę kolorowań z użyciem ustalonych liczb
  poszczególnych kolorów. Zbiór kolorów $K$ potraktujemy jako zb. zmiennych
  formalnych $\set{t_1,\ldots,t_k}$. Definiujemy wagę kolorowania
  $f: w(f) = t_1^{a_1}\cdot\ldots\cdot t_k^{a_k}$, jeśli $f$ nadaje kolor $t_i$
  dokładnie $a_i$ elementom.

Tw. (Polya): Enumeratorem stanowiącym sumę wag kolorowań nierównoważnych wzgl.
  $G$ jest $I_G(\sum_{t\in K}t, \sum_{t\in K}t^2, \ldots, \sum_{t\in K}t^n)$
