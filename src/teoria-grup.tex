\section{Teoria grup}

$G = \angles{e, \cdot}$: łączność, elem. neutr., odwrotność, przemienność*;

Podgrupa grupy $G$, to podzb. zamkn. na $\cdot$ oraz $^{-1}$;

Podgr. generowana przez $A$, to przecięcie wszystkich podgr. $G \supseteq A$;

$G(A)$ składa się ze skończonych iloczynów $g_1\dots g_k$,
  $g_i\in A \lor g_i^{-1} \in A$;

Gr. cykliczna, czyli generowana przez 1 elem.;

Gr. cykliczna $\Leftrightarrow$ przemienna;

Każda podgr. gr. cyklicznej jest cykliczna;

Gr., w której każdy elementy $\neq e$ ma rząd $2$, jest przemienna i ma rząd $2^k$;

% Twierdzenie (Lagrange).
Tw. (Lagrange): Rząd podgr. $H$ gr. skończonej $G$ dzieli rząd $G$;

$G$ cykliczna ma $\phi(n)$ generatorów;

Tw.: $s(d)$, to l. elem. rzędu $d$ w $G$. $s(d) =$ if $d \not{|} n$ then $0$
  else $\phi(d)$;

% Twierdzenie (Cayley).
Tw. (Cayley): $S(Y)$, to zbiór bijekcji $Y \rightarrow Y$ ze składaniem
  ($S_n$ dla $Y=\set{1,\dots,n}$). Każda gr. rzędu $n$ jest izomorf. z pewną
  podgr.  $S_n$;

% TODO: Dokończyć teorię grup. Slajd 216.
