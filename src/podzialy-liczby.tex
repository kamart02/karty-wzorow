\section{Podział liczby / enumerator podziałów}

Podział liczby $n$, to przedstawienie $n$ jako sumy nierosnących dod. skład.;

L. podz. $n$ na $\leq k$ skład. $\leftrightsquigarrow$
  l. podz. $n+k$ na dokł. $k$ skład.;

Enum. podziałów $P(x) = \sum_nP_nx^n =
  (1+x+x^2+\dots)\cdot(1+x^2+x^4+\dots)\dots(1+x^k+\dots)\dots =
  \frac{1}{(1-x)(1-x^2)\dots(1-x^k)\dots}$, gdzie wybranie z $k$-tego nawiasu
  $x^{ik} \leftrightsquigarrow$ wzięciu do podziału $i$ razy składnika $k$;

Enum. podz. o składnikach $\leq k$: $p_{\leq k} (x) =
  \frac{1}{(1-x)(1-x^2)\dots(1-x^k)}$;

Enum. podz. 1 zł na grosze:
  $\left [ x^{100} \right ]
  \frac{1}{(1-x)(1-x^2)(1-x^5)(1-x^{10})(1-x^{20})(1-x^{50})}$;

Enum. podz. na składniki $>1$: $\frac{1}{(1-x^2)(1-x^3)\dots}=(1-x)P(x) =
  P_n - P_{n-1}, n>0$;

Enum. podz. o różnych s.: $r(x) = (1+x)(1+x^2)(1+x^3)\dots$;

Enum. podz. o nieparz. s.: $n(x)=\frac{1}{(1-x)(1-x^3)\dots}$;

Tw.: $\#$ podz. $n$ na różne s. $= \#$ podz. na nieparz. s.: $n(x) = r(x)$;

% TODO: Przykład użycia.
Toż. Eulera: $nP_n = \sum_{k=0}^{n-1}\sigma(n-k)P_k$,
  gdzie $\sigma = \sum_{k|n}k$;
