\section{Teoria liczb}

% NWD.
\entry
$NWD(a,b) = min_{+}\set{ax+by | x,y\in \mathbb{Z}}$;
\entry
$a \perp b \Leftrightarrow NWD(a,b)=1$;

% Algorytm Euklidesa.
\entry
Alg. Euklidesa: $NWD(a,b)=ax+by$. Jeśli $b=0$, to
  $\angles{x,y} \angles{1,0}$, wpp
  $NWD(a,b) = NWD(b, a \mod b) = bx' + (a \mod b)y'$
  ($\angles{x,y} \leftarrow \angles{y', x' - y'\cdot\lfloor{a/b}\rfloor}$);

% Tabelka z przykładem działania algorytmu Euklidesa.
% TODO: Dodaj ramkę.
\entry
\begin{tabular}{c c c c}
    20& 56 & 3\tikzmark{aedst3} &-1 \\
    16& 20 & -1\tikzmark{aedst2} &\tikzmark{aesrc3}1 \\
    4 & 16 & 1\tikzmark{aedst1} &\tikzmark{aesrc2}0 \\
    0\tikzmark{aesrc0} & 4 &\tikzmark{aedst0}0 & \tikzmark{aesrc1}1 \\
\end{tabular}
\begin{tikzpicture}[overlay, remember picture, yshift=.25\baselineskip,
    shorten >=.5pt, shorten <=.5pt]
    \draw [->] ({pic cs:aesrc0}) [bend left] to ({pic cs:aedst0});
    \draw [->] ([yshift=.75pt]{pic cs:aesrc1}) -- ({pic cs:aedst1});
    \draw [->] ([yshift=.75pt]{pic cs:aesrc2}) -- ({pic cs:aedst2});
    \draw [->] ([yshift=.75pt]{pic cs:aesrc3}) -- ({pic cs:aedst3});
\end{tikzpicture}

% Podstawowe wiadomości z teorii liczb.
\entry
$a = \prod^m_{i=1}p_i$;

\entry
${a | bc \land a \perp b} \Rightarrow {a | c}$

\entry
${NWW(a,b)} = {ab/NWD(a,b)} = {\prod^k_{i=1}p_i^{max(\alpha_i, \beta_i)}}$;

\entry
${a \equiv_{n} b}$ oraz ${c \equiv_{n} d} \implies
  {a+c\equiv_n b+d}$ oraz ${a\cdot c \equiv_n b\cdot d}$;

\entry
${d \perp n} \land {ad \equiv bd \pmod{n}} \Rightarrow {a \equiv b \pmod{n}}$;

\entry
${ad \equiv bd \pmod{nd}} \Leftrightarrow {a\equiv b \pmod{n}}$;

\entry
${b = a^{-1} \pmod{n}} \Leftrightarrow {ab \equiv 1 \pmod{n}}$;

\entry
${n \perp m} \implies
  {\left( a\equiv b \pmod{n} \land a \equiv b \pmod{m} \Leftrightarrow
  a\equiv b \pmod{nm} \right)}$ (ugogólnia się na dow. liczbę parami wzgl.
  pierwszych modułów $n_1,\ldots,n_k$);

\entry
$a \perp n \Rightarrow a^x = a^{x \mod \phi(n)} \mod n$;

% Chińskie twierdzenie o resztach.
\framebox{\vbox{
  \entry
    CRT: ${n = n_1\dots n_k}, {n_i \perp n_j}$, to
      $\forall_{a_1, \dots, a_k}\exists a\in\set{0,\dots,n-1}$ t., że
      $a\equiv a_i \pmod{n_i}$ dla $i=1,\dots,k$;

    % Przykład użycia chińskiego twierdzenia o resztach.
    % Źródło: wazniak.mimuw.edu.pl:Matematyka dyskretna 1: Wykład 11
  \entry
    \textsubentry{1}
    $x \equiv_{3} 2$, $x \equiv_{10} 7$, $x \equiv_{11} 10$,
      $x \equiv_{7} 1$;
      \textsubentry{2}
      Warunki CRT:
      $\forall_{n_i, n_j, n_i \neq n_j} NWD(n_i, n_j) = 1$
      \textsubentry{3}$n = 3\cdot 7\cdot 10\cdot 11 = 2310$,
      $m_1 = \frac{2310}{3} = 770$,
      $m_2 = \ldots = 231$,
      $m_3 = \ldots = 210$,
      $m_4 = \ldots = 330$;
      \textsubentry{4}$m_1^{-1} \mod n_1 = 1$, $\ldots$ oraz
      $x_1 = m_1 (m_1^{-1} \mod n_1)$, $\ldots$
      \textsubentry{5} R. alg. Eukl.:
      $NWD(3, 770) = 1 = 257\cdot 3 - 1\cdot 770 \Rightarrow
        x_1=-1\equiv_{3} 2$,
      $NWD(10, 231) = \ldots \Rightarrow x_2=1$,
      $NWD(11, 210) = \ldots \Rightarrow x_3 = 1$,
      $NWD(7, 330) = \ldots \Rightarrow x_4 = 1$;
      \textsubentry{6}
      $x = 2\cdot 2 \cdot 770 + 7\cdot 1\cdot 231 + 10\cdot 1\cdot 210 +
      1\cdot 1\cdot 330$;

  \entry
      Ogólniej: $a^{\phi(p_1^{\alpha_1})} \equiv 1 \pmod{p_1^{\alpha_1}}$,
      $a^{\phi(p_2^{\alpha_2})} \equiv 1 \pmod{p_2^{\alpha_2}} \Rightarrow
      a^{NWW(\phi(p_1^{\alpha_1}),\phi(p_2^{\alpha_2}))} \equiv 1 \pmod{n}$,
      gdzie $n=p_1^{\alpha_1}p_2^{\alpha_2}$;

  \entry
    $x^{118} \equiv 113 \pmod{1001}$;
    \textsubentry{1}
    $x^{118} \equiv_7 1$, $\ldots\equiv_{11} 3$, $\ldots\equiv_{13} 9$ są $\perp$,
      więc tw. Eulera;
    \textsubentry{2}
    $x^2 \equiv_7 1^{-1}$, $x^2 \equiv_{11} 3^{-1}$, $x^2 \equiv_{13} 9^{-1}$
    \textsubentry{3}
    $x^2 \equiv_7 1$, $x^2 \equiv_{11} 4$, $x^2 \equiv_{13} 3$
    \textsubentry{4}
    $x \equiv_7 \pm 1$, $x \equiv_{11} \pm 2$, $x \equiv_{13} \pm 4$
    \textsubentry{5}
    Teraz z CRT: $\ldots$: $c_1 = 5 \cdot 143$, $c_2 = 4\cdot 91$,
      $c_3 = 12\cdot 77$;
}}

% Małe twierdzenie Fermata.
\entry
MTF: ${p\in \mathbb{P} \land p\not | a} \Rightarrow a^{p-1} \equiv 1 \pmod{p}$,
  inaczej $a^p \equiv a \pmod{p}$;

% Funkcja Eulera.
\entry
F. Eulera: $\phi: \mathbb{N} \rightarrow \mathbb{N}$ t., że
  $\phi(n)=|\mathbb{Z}^*_n|=|\set{1\leq k\leq n : k \perp n}|$;

\entry
$p \in \mathbb{P} \Rightarrow \phi(p^k)=p^k-p^{k-1}$;

\entry
$m\perp n \Rightarrow \phi(mn) = \phi(m)\phi(n)$, np.:
  $\phi(2016) = \phi(2^3)\phi(3^2)\phi(7) =
  2016(1-\frac{1}{2})(1-\frac{1}{3})(1-\frac{1}{7})$;

\entry
$\phi(n) = n\prod_{\mathbb{P} \ni p | n}(1- 1/p)$;

\entry
$\sum_{d|m}\phi(d) = m$;

% Twierdzenie Eulera.
\entry
Tw. Eulera: $a\perp n \Rightarrow a^{\phi(n)} \equiv 1 \pmod{n}$;

% Twierdzenie Wilsona.
\entry
Tw. Wilsona: $p\in\mathbb{P} \Leftrightarrow (p-1)! \equiv -1 \pmod{p}$;

% RSA.
\entry
Alg. RSA: $n=pq, e, M, e \perp \phi(n)$. Wtedy $E(M) = M^e \mod n$,
  $d = e^{-1} \mod \phi(n)$, $D(C) = C^d \mod n$. Działa, bo
  $(M^e)^d \mod n = \text{z tw. Eulera} = M^{ed \mod n} = M^1 \mod n$;

% Test Millera-Rabina.
\entry
Test Millera-Rabina: $\exists_{0<a<n}a^{n-1}\not\equiv 1 \pmod{n} \Rightarrow
  n \not\in \mathbb{P}$ (słaby, bo liczby Carmichaela);

% TODO: Opis liczb Carmichaela.

% TODO: Faktoryzajca metodą Fermata.

% TODO: Faktoryzacja metodą "p-1" Pollarda.
