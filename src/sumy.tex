\section{Sumy}

${S_n = \sum^{n}_{k = 0} a \cdot x^k} = {a \frac{1 - x^{n+1}}{1 - x}}$;

$\sum_{i=0}^{n-1}\frac{1}{(i+1)(i+2)} = \frac{n}{n+1}$;

$\sum_{i=0}^{n}(-1)^i i^2 = (-1)^n\frac{n(n+1)}{2}$;

\subsection{Sumowanie przez części}

${\Delta f(x) = f(x + 1) - f(x)}$;
${\Delta x^{\underline{n}}} = {n x^{\underline{n - 1}}}$;
${\mathcal{S}^{b}_{a} f}={g |^b_a} = {\sum^{b - 1}_{k = a} f(k)}$;
${\Delta (rt)} = {r \Delta t + E t \Delta r \text{, gdzie } E t(x)} =
  {t(x + 1)}$;
${\mathcal{S}^{b}_{a} r \Delta t}= {r t |^b_a - \mathcal{S}^b_a E t \Delta r}$;

\subsection{Dwumian}

${\binom{n}{k} = {\frac{n^{\underline{k}}}{k!}}}$;
${\sum_{i = 0}^n \binom{n}{i} = 2^n}$;
${\sum_{i = 0}^{n} (-1)^i \binom{n}{i} = [ n = 0 ]}$;
$\binom{n}{k} = \binom{n}{n - k}$;

$\binom{n}{k} = \frac{n}{k} \binom{n - 1}{k - 1}$;

${(x + y)^n = \sum_{i \geq 0} \binom{n}{i} x^i y^{n-i}, n \in \mathbb{N}}$;

${(1 + x)^r = \sum_{i \geq 0} \binom{r}{i} x^i, r \in \mathbb{R}, |x| < 1}$;

$(x + y + z)^n =
  \sum_{0 \leq a, b \leq n, a+b \leq n} \binom{n}{a,b} x^a y^b z^{n-a-b}$;

$\binom{n}{a,b} = \binom{n}{a}\binom{n-a}{b} = \frac{n!}{a! b! (n - a - b)!}$;

$\binom{n}{k}\binom{k}{i} = \binom{n}{i}\binom{n-i}{k-i}$;

$\binom{n}{k} = \begin{cases}
    1, k = 0 \text{ lub } k = n \\
    \binom{n - 1}{k} + \binom{n - 1}{k - 1}, 0 < k < n
\end{cases}
$;

% Tożsamość Cauchy'ego.
Toż. (Cauchy): $\sum^k_{i=0}\binom{n}{i}\binom{m}{k - i} = \binom{n + m}{k}$;

$\text{$\sum$ równoległe: } \sum^k_{i=0} \binom{y + i}{i} =
  \binom{y + k + 1}{k}$;

$(-1)^i\binom{x}{i} = \binom{i - 1 - x}{i} \text{, bo }  x^{\underline{i}} =
  (-1)^i(i - 1 - x)^{\underline{i}}$;

$a_n = \sum_i\binom{n}{i}(-1)^i b_i \iff b_n = \dots$;

% Inwersje.
\subsection{Inwersje}

$\Pi = 1^{\lambda_1}2^{\lambda_2}\dots n^{\lambda_n}$, to $sgn(\Pi) =
  (-1)^{\Sigma_i\lambda_{2i}}$;

% Liczby Stirlinga I rodzaju.
\subsection{Liczby Stirlinga}

${n \brack k}$ - podziały $n$-zbioru na $k$ cykli (kolejność w cyklu istotna z
  dokładnością do cyklicznego przesunięcia);

${n \brack 0} = [ n = 0 ]$;
${n \brack 1} = (n - 1)!, n > 0$;
$\sum_k{n \brack k} = n!$;

${n \brack k} = (n-1){n-1 \brack k}+{n-1 \brack k-1}$ dla ${k > 0}$;

% Liczby Stirlinga II rodzaju.
${n \brace k}$ - podziały $n$-zbioru na $k$ bloków;

${n \brace 0} = [n = 0]$;
${n \brace 1} = 1$;
${n \brace 2} = 2^{n-1}-1$;

${n \brace k} = k{n-1 \brace k} + {n-1 \brace k-1}$ dla ${k > 0}$;

% Własności liczb Stirlinga.
$x^n = \sum_k{n \brace k} x^{\underline{k}}$;

$x^{\overline{n}} = \sum_k{n \brack k}x^k$;

$(-x)^{\overline{k}} = (-1)^kx^{\overline{k}}$;

$x^n = \sum_k {n \brace k} (-1)^{n-k} x^{\overline{k}}$;

$x^{\underline{n}} = \sum_k {n \brack k} (-1)^{n-k} x^{k}$;

$x^n = \sum_{i,k} {n \brace i}{i \brack k}(-1)^{n-i}x^k$;

$\sum_i{n \brace i}{i \brack k}(-1)^{n-i} = [n=k] =
  \sum_i{n \brack i}{i \brace k}(-1)^{n-i}$;

$a_n = \sum_i{n \brace i}(-1)^ib_i \iff b_n = \sum_i{n \brack i}(-1)^ia_i$;
